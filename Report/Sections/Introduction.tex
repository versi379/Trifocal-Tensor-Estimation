\section{Introduction}\label{sec:intro}
Since computer vision's inception, the study of cameras and images has been a central focus of the field's efforts. At its heart are complex processes like determining positions and reconstructing 3D shapes, both heavily dependent on understanding the complex relationship between points in space and how they appear in images, based on the principles of perspective projection in pinhole cameras. This understanding paves the way for triangulating spatial points from their corresponding image projections.\\

Embedded within this framework, the fundamental matrix emerges as a pivotal algebraic entity, encoding the essential correspondence between matching image points. It serves as a pathway for understanding the relative positions and orientations of pairs of camera viewpoints, essential for various applications in computer vision. Broadening this framework to include three perspectives introduces the concept of the trifocal tensor, a mathematical construct that encapsulates the algebraic constraints governing the relationships among three corresponding image points, often referred to as trilinearities. Although theoretically plausible to derive a multi-view matrix accommodating an arbitrary number of views, practical constraints predominantly stem from pairs or triplets of views. Consequently, most multi-view structure-from-motion pipelines pivot around initial view pairs or triplets for practical implementation.\\

Traditionally, the trifocal tensor has been preferred over the fundamental matrix when dealing with a triplet of views. We undertake the task of challenging this established preference by conducting an in-depth investigation into the comparative performance of the trifocal tensor versus that of the fundamental matrix.\\

In Section (\ref{sec:tft}), we meticulously define and parameterize the trifocal tensor, delving into its intricacies. Subsequently, Section (\ref{sec:estimation}) explores the techniques employed for its estimation and subsequent pose determination. Our journey culminates in Section (\ref{sec:experiments}), where we present empirical findings quantifying the performance of both methodologies. These results, analyzed in Section (\ref{sec:conclusions}), lead us to the conclusion that while the trifocal tensor does offer certain advantages, they are not substantial enough to unequivocally declare it superior to the fundamental matrix.

\subsection{Notation}
In this paper, we adopt specific notation conventions: vectors are denoted by lowercase (\( v \)), matrices by uppercase (\( M \)), tensors by calligraphic bold uppercase (\( \mathbfcal{T} \)), and tensors' correlation slices (\ie, matrices) by bold uppercase (\( \bm{T}_i \)).\\

The \( 3 \times 3 \) matrix representation of the cross product with a 3-vector $v$ is indicated by \( [v]_{\times}w \), \ie, \( [v]_{\times}w = v \times w \), where \( w \) represents any given vector.\\

The \( L^2 \) norm of a vector \( v \) is denoted as \( \Vert v \Vert \), while for matrices or tensors, it represents the \( L^2 \) norm of the vector constructed from their coefficients. The Frobenius norm of a matrix \( M \) is denoted as \( \Vert M \Vert \), while for a tensor \( \mathbfcal{T} \), it signifies the square root of the sum of squares of all its elements, denoted as \( \Vert \mathbfcal{T} \Vert \coloneqq \sqrt{\sum_{i,j,k} (\bm{T}_{i}^{jk})^2} \). Additionally, \( \vert M \vert \) refers to the determinant of matrix \( M \).
